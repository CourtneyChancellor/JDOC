\documentclass[en]{jdoc}

\usepackage[charter]{mathdesign}

\usepackage[colorlinks=true,urlcolor=blue]{hyperref} % hypertext links in color

\ed{PRES LUNAM\\École Doctorale STIM\\Sciences et Technologies de\\l'Information et Mathématiques}
\spec{Specialty}
\labo{Lab}
\equipe{Team}

\title{Information and formatting of JDOC articles}
\author{Author’s name}
\email{first.last@lab-city.edu}

\begin{document}

\makehead % builds header (ed, specialty, labo, title, author, ...).

\begin{abstract}
  This paragraph lists the constraints you have to follow when write
  an abstract for the Annual Day of ED STIM.  It also gives you an
  illustration of the set visual style.  \textbf{In addition to your
    article, you must write an abstract of your research work.  Your
    abstract must be between 800 and 1500 characters, or between 8 to
    15 lines.  All submitted abstracts will be consolidated into the
    Handbook of Abstracts}.  The abstract’s content must be as
  accessible as possible.  The JDOC organizing committee reminds you
  to strictly follow these length constraints, and to submit your
  abstract along with your article before the due date.
\end{abstract}

\begin{keywords}
Must be italicized, JDOC, format, size, page layout, dates.
\end{keywords}

\begin{collaborations}
Academic research organisms, industry partners, work groups.
\end{collaborations}

\section{JDOC}
All articles submitted to JDOC will be read by members of the
organizing committee.  Students can volunteer to do an oral
presentation, but JDOC committee will ultimately decide who gets a
spot, depending on the submitted articles.  All articles will be
gathered into proceedings, which you will receive on the day of the
conference.  The goal of these proceedings is to widen the knowledge
of PhD students by getting them to discover and understand what
happens in other domains of the ED STIM.  Therefore, each article must
present a student’s work with a focus on accessibility.

\subsection{Technical information}
Your article must be laid out on A4 paper, in one column, and a
maximum length of four pages (including figures, tables and
references).  It must be written in French or English (mind your
spelling).

The article must contain: an abstract (15 lines long at most),
keywords, a content, a conclusion and bibliographical references.  The
following must also appear in the header: doctoral school, doctoral
specialty\footnote{As a reminder, the doctoral specialties of ED STIM
  are: Automatique, Robotique, Traitement du Signal ; Mathématiques ;
  Informatique ; Systèmes Électroniques et de Télécommunications,
  Génie Électrique ; Systèmes de Production.}, lab and research team,
and your name.

\subsection{Deadline}
We will not accept any article submitted after the paper submission
deadline.  If you know you will have difficulties meeting the
deadline, please contact us beforehand at \url{jdoc@univ-nantes.fr}.

\subsection{File name}
Your submitted article must be named in the following fashion:
“\textit{name}\_\textit{lab}\_papier.pdf”, where
%
\begin{description}
\item \textit{name}: your name
\item \textit{lab}: acronym of your lab
\end{description}
%
Example: “DUPONT\_IRCCyN\_papier.pdf”.

\section{Article layout with \LaTeX{}}
\LaTeX{} users should use the file “jdoc.cls” to comply with the JDOC
submission instructions; no extra effort required.  New commands are
defined to simplify formatting your article.  This file defines a new
document class \LaTeXe{} that you must include in your document.

\subsection{How to include the JDOC document class}
%
\begin{enumerate}
\item Copy the file “jdoc.cls” into the directory containing your
  “.tex” article.
\item Include the \verb|jdoc| class like so:
%
\begin{verbatim}
\documentclass[options]{jdoc}
\begin{document}
...
\end{document}
\end{verbatim}
%
The optional parameter \verb|options| can take the usual values
(\verb|a4paper|, \verb|10pt|, \dots). Depending on the language you
want to write your article in (English or French), you should add
\verb|en| or \verb|fr|.  For instance:\\

\verb|\documentclass[en]{jdoc}| $\Longrightarrow$
\framebox{{\small {\bf Abstract:} blather .., {\bf Keywords:} blather ..}}\\

\verb|\documentclass[fr]{jdoc}| $\Longrightarrow$
\framebox{{\small {\bf Résumé~:} blather .., {\bf Mots clés~: } blather ..}} \\

\end{enumerate}

\subsection{Additional commands provided by the jdoc class}
The additional commands to simplify writing your article are:

\begin{verbatim}
\ed{..}                              % name of the doctoral school
\spec{..}                            % specialty
\labo{..}                            % lab
\equipe{..}                          % research team
\title{..}                           % article title
\author{..}                          % your name
\email{..}                           % your email address

\makehead                            % prints the header (ed, spec,
                                     % lab, author, title, ...)
\begin{abstract}...\end{abstract}
\begin{keywords}...\end{keywords}
\end{verbatim}

\subsection{Preloaded packages}
The following packages are loaded by the jdoc document class:
%
\begin{verbatim}
\usepackage{amsmath,amsfonts}           % typeset math formulas
\usepackage{mathrsfs,amssymb}           % extra math symbols
\usepackage[utf8]{inputenc}             % accented characters (é,â,..)
\usepackage[T1]{fontenc}                % accented characters (é,â,..)
\usepackage{graphicx}                   % inclusion of images
\usepackage{multicol}                   % multi-column support
\usepackage{color}                      % colored text
\usepackage[french]{babel}              % for writing in French
\usepackage[english]{babel}             % or if we write in English
\end{verbatim}
%
For additional information on \LaTeXe{}, consult \cite{Oet06}.

\section{Section breakdown}
\begin{itemize}
\item The abstract must follow the one in this document; no longer than
  \textbf{15 lines}.

\item Keywords must be italicized.

\item Recall that your article will be read by PhD students from other
  domains.  You must thrive to be the as accessible as possible,
  without losing the scientific fabric of your work.  Use images with
  parsimony.

\item Bibliographical references will be listed in the order they
  appear in the article.  They will appear numbered like so:
  \cite{Oet06}\cite{Mau84}\cite{max82}, etc.

\end{itemize}

\bibliography{biblio-en}

%Ou utiliser bibitem
%\begin{thebibliography}{3}
%   \bibitem[1]{Mau84} J.R. MAUTZ and R.F. HARRINGTON, "An E-Field solution for a conducting surface small or comparable to the wavelength", IEEE Trans.  AP, vol.  32, pp.  330-339, 1984.
%   \bibitem[2]{max82} J.CLERCK MAXWELL, Atreatise on Electricity and Magnetism, 3rd ed., vol.  2. Oxford : Clarendon, 1982.
%   \bibitem[3]{Oet06} TOBIAS OETIKER {\it et al.}, "Une courte introduction à \LaTeXe{}", disponible à l'adresse
   %\emph{\href{http://www.ctan.org/tex-archive/info/lshort/french/flshort-3.20.pdf}{http://www.ctan.org/tex-archive/info/lshort/french/flshort-3.20.pdf}}.
%\end{thebibliography}

\end{document}
