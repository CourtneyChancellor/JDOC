\documentclass[en]{jdoc}

\ed{PRES LUNAM \\ École Doctorale STIM \\ Sciences et Technologies de l'Information
et Mathématiques}
\spec{BioInformatics}
\labo{IRCCyN}
\equipe{MeForBio}
\title{Qualitative Modeling of Gene Regulatory Networks}
\author{Courtney Chancellor}
\email{Courtney.Chancellor@ec-nantes.fr}


\begin{document}

\makehead
 
 \begin{abstract}
 A large part of bioinformatics is dedicated to the modeling of biological systems with the hopes of supplementing experimental work or informing experimental design. Many modeling frameworks exist, each of which brings its own strengths and weaknesses, data requirements and fundamental assumptions. As a result, what kind of framework to choose is largely informed by the characteristics of the system or the nature of the information we wish to derive from our model. For regulation of genetic expression, qualitative modeling schemes can be advantageous as they are well suited for handling the very large, interconnected graph of interacting species typical of these systems. However, in exchange for being scalable, qualitative models typically give limited access to the underlying probability distribution of the state space, normally requiring simulation techniques. In our work, we attempt to apply numerical methods to qualitative systems in order to directly access this probability distribution, deepening our potential to analyze very large, or high dimensional, systems.
\end{abstract}

\begin{keywords}
 genetic regulation, 
 qualitative modeling, 
 high dimensional systems, 
 numerical methods
\end{keywords}

\begin{collaborations}
Francisco Chinesta,
Morgan Magnin,
Olivier Roux
\end{collaborations}

\section{Introduction}

 \section{}
 
 
\end{document}
